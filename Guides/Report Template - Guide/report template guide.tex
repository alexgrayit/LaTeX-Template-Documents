\documentclass[a4paper, 12pt, english]{article}
% -------------------------------------------------------------
%   ------ See bottom of document to edit parameters -------  |
% -------------------------------------------------------------




% ---------------------------------------------------------------------------------------------------
% || ------------------------------------ LOTS OF PACKAGES --------------------------------------- ||
% ---------------------------------------------------------------------------------------------------

% OK just a warning - PACKAGE ORDER MATTERS - for instance the maths packages
% not being at the top breaks the compilation. LaTeX shenanigans can be annoying.


% Maths Formatting Packages - they just add extra commands
\usepackage{amsmath}
\usepackage{array} % extends array formatting
\usepackage{amssymb}
\usepackage{nccmath}
\usepackage{esint} % more integrals
\usepackage{bm} % bold math text that's not awful
\usepackage{xfrac} % slanted frac \sfrac
\usepackage{bigints}
\usepackage{tabularx}
\usepackage{cancel}
\usepackage{mathtools}%idk whether this is needed but ¯\_(ツ)_/¯



% Document/Text Formatting Packages
\usepackage[a4paper, portrait, right=20mm, left=20mm, top=25mm, bottom=20mm, showframe=false]{geometry} % margins
\usepackage{booktabs} % tables - needed so that tables from latex table makers work
\usepackage{babel} % language options
\usepackage[utf8]{inputenc} % adds extra unicode characters
    \DeclareUnicodeCharacter{2212}{-}
\usepackage{dtk-logos}
\usepackage{multicol}
\usepackage{paracol}
\usepackage{fix-cm}
\usepackage{import}
\usepackage{csquotes} % it told me to
\usepackage{xcolor} % colours yay
\usepackage{xpatch}
\usepackage{accents} % text accents
\usepackage[none]{hyphenat} % hyphenation
\usepackage{tocloft} % table of contents control
    \renewcommand{\cftsecleader}{\cftdotfill{\cftdotsep}} % adds dots between section titles and page numbers in ToC
\usepackage[justification=centering, margin=10mm]{caption} 
\usepackage{subcaption} % improves captions for subfigures


\usepackage{placeins} % Adds the \FloatBarrier command
\usepackage[hidelinks, plainpages=false, pdfpagelabels]{hyperref} % Hyperlinks - yay we live in the modern world
\usepackage{url} % adds url linking
\usepackage{nameref} % adds ability to reference sections by name
\usepackage{bookmark}

\usepackage{calc} % adds \setlength and other commands
\usepackage{lastpage} % page number referencing

\usepackage[bottom]{footmisc} %footnote customizability, [bottom] forces footnotes to the bottom

\usepackage{fancyhdr} % fancy headers and footers

\usepackage[T1]{fontenc} 
\usepackage{lmodern} % fixes some font issues that happen sometimes
\usepackage{titling} % title stuff
\usepackage{titlesec} % title format customisation


\usepackage{abstract} % Allows abstract customization
    \addto{\captionsenglish}{\renewcommand{\abstractname}{}}   % clear the title
    \renewcommand{\absnamepos}{empty} % originally center -- don't want an abstract title or it's whitespace


% Physics Packages
\usepackage[arrowdel]{physics} % - \grad{} \div{} \curl{} \laplacian{}
\usepackage{siunitx} % adds si units \si{\unit} (it's hard to use at first, look up the documentation)


% Code Formatting Packages
\usepackage{listings} % code text environment
\usepackage{verbatim}


% Graphics Formatting Packages
\usepackage{tikz}
\usepackage{tikzscale}
    \usetikzlibrary{shapes,arrows,positioning}
    \usetikzlibrary{matrix,calc}
\usepackage{tikz-3dplot}
    \usetikzlibrary{calc,patterns,angles,quotes}
\usepackage{graphicx}
\usepackage{graphics}

\usepackage{pgfplots}
    % Dunno, it complains if this isn't here
    \pgfplotsset{compat=1.17}
\usepackage{pgf} % allows inclusion of pgf vector graphic images
\usepackage{adjustbox}




% ---------------------------------------------------------------------------------------------------
% || --------------------------------------- REFERENCING ----------------------------------------- ||
% ---------------------------------------------------------------------------------------------------

\usepackage[backend=biber, style=ieee]{biblatex}
\addbibresource{references.bib}% Syntax for version >= 1.2
% the references.bib should be in the same directory as the tex file





% ---------------------------------------------------------------------------------------------------
% || --------------------------------------- TEXT SPACING ---------------------------------------- ||
% ---------------------------------------------------------------------------------------------------

% overridden in customization at bottom
\setlength{\parindent}{0em}
\setlength{\parskip}{1em}


\newlength{\tablegap}
\setlength{\tablegap}{0.3em} % put in square brackets after i.e. \\[\tablegap]



% ---------------------------------------------------------------------------------------------------
% || ---------------------------------- CUSTOM MATHS COMMANDS ------------------------------------ ||
% ---------------------------------------------------------------------------------------------------


\setlength{\jot}{0.5em} % length between newlines of align environment

% Symbols and Notation
\newcommand{\ud}{\ \mathrm{d}}
\renewcommand{\d}{\mathrm{d}}
\newcommand{\Reals}{\mathbb{R}}
\newcommand{\Complexs}{\mathbb{C}}
\newcommand{\Integers}{\mathbb{Z}}
\newcommand{\Naturals}{\mathbb{N}}
\newcommand{\Rationals}{\mathbb{Q}}
\newcommand{\Primes}{\mathbb{P}}

\newcommand{\emf}{\mathcal{E}}

\newcommand{\deq}{\coloneqq} % defined as equal

% Size overriding - big things
\newcommand{\bfrac}[2]{\frac{\displaystyle #1}{\displaystyle #2}}
\newcommand{\bint}{\displaystyle \int}

% Vector Formatting
\newcommand{\hvec}[1]{\accentset{\rightharpoonup{}\vspace{-0.5mm}}{#1}} % harpoon above
\newcommand{\svec}[1]{\underaccent{\tilde}{#1}} % squiggle below
\newcommand{\bvec}[1]{\boldsymbol{\mathbf{#1}}} % math bold font

% -Long harpoon vector
\makeatletter
\newcommand*{\rightharpoonupfill@}{\arrowfill@\relbar\relbar\rightharpoonup}
\newcommand{\lhvec}{\mathpalette{\overarrow@\rightharpoonupfill@}}
\newcommand{\lvect}[1]{\bvec{\lhvec{#1}}}%final edit here
% -end Long harpoon vector

\newcommand{\vect}[1]{\bvec{\hvec{#1}}} % used vector format
\newcommand{\vhat}[1]{\bvec{\hat{#1}}} % used hat format

\newcommand{\vmod}[1]{\left\| #1 \right\|} % modulus notation

% Vector Calculus Operators
\renewcommand{\grad}{\vect{\nabla}}
\renewcommand{\div}{\vect{\nabla} \cdot}
\renewcommand{\curl}{\vect{\nabla} \times}
\renewcommand{\laplacian}{\bvec{\nabla}^2}
\newcommand{\del}{\vect{\nabla}}

% projection operator
\DeclareMathOperator{\vectProj}{proj}
\newcommand{\proj}[2]{\vectProj_{#2} {#1}}

% Custom Matrix - better spacing
\newcommand{\Matrix}[1]{\renewcommand{\arraystretch}{1.25}
    \left[\begin{matrix}
    #1
\end{matrix}\right]
\renewcommand{\arraystretch}{1}}




% ---------------------------------------------------------------------------------------------------
% || ------------------------------- INTERNAL REFERENCING COMMANDS ------------------------------- ||
% ---------------------------------------------------------------------------------------------------


\newcommand{\secrefNumName}[1]{\mbox{\hyperref[#1]{\S}\ref{#1}\hyperref[#1]{:\hspace{1pt}}\nameref{#1}}} % ref with Num:Name
\newcommand{\secrefName}[1]{\mbox{\hyperref[#1]{\S\hspace{1pt}}\nameref{#1}}} % ref with just name
\newcommand{\secrefNum}[1]{\mbox{\hyperref[#1]{\S}\ref{#1}}} % ref with just num

\newcommand{\secref}[1]{\secrefNumName{#1}} % choose method of section reference


\newcommand{\eqnum}[1]{\ref*{#1}}


% Lower case commands
\newcommand{\appref}[1]{\mbox{\hyperref[#1]{appendix\hspace{2pt}}\ref{#1}}}
\renewcommand{\eqref}[1]{\mbox{\hyperref[#1]{equation\hspace{2pt}}\ref{#1}}}
\newcommand{\figref}[1]{\mbox{\hyperref[#1]{figure\hspace{2pt}}\ref{#1}}}
\newcommand{\tableref}[1]{\mbox{\hyperref[#1]{table\hspace{2pt}}\ref{#1}}}

% Capitalised Commands
\newcommand{\Appref}[1]{\mbox{\hyperref[#1]{Appendix\hspace{2pt}}\ref{#1}}}
\newcommand{\Eqref}[1]{\mbox{\hyperref[#1]{Equation\hspace{2pt}}\ref{#1}}}
\newcommand{\Figref}[1]{\mbox{\hyperref[#1]{Figure\hspace{2pt}}\ref{#1}}}
\newcommand{\Tableref}[1]{\mbox{\hyperref[#1]{Table\hspace{2pt}}\ref{#1}}}


% use \cite and \textcite to do references




% ---------------------------------------------------------------------------------------------------
% || --------------------------------- HYPERLINK COLOURING --------------------------------------- ||
% ---------------------------------------------------------------------------------------------------

% overridden by customization below
\usepackage{hyperref,xcolor}
\definecolor{darkblue}{rgb}{0,0,0.6}
\hypersetup{
    %pdfpagemode=FullScreen,
    colorlinks=true,
    linkcolor=darkblue,
    filecolor=magenta,      
    urlcolor=blue,
    citecolor=black
}




% ---------------------------------------------------------------------------------------------------
% || ---------------------------------- SECTION FORMATTING --------------------------------------- ||
% ---------------------------------------------------------------------------------------------------

\setlength{\droptitle}{-8em}


%spacing is:  left, before, after
\titlespacing{\section}{0pt}{1em}{-0.5em}
\titlespacing{\subsection}{0pt}{1em}{-0.7em}
\titlespacing{\subsubsection}{0pt}{1em}{-0.8em}
\titlespacing*{\section}{0pt}{5.5ex plus 1ex minus .2ex}{2pt}

\usepackage{bold-extra}
%\titleformat{<command>}[<shape>]{<format>}{<label>}{<sep>}{<before-code>}[<after-code>]
\titleformat{\section}{\normalfont\Large\bfseries}{\thesection}{1em}{}
\titleformat{\subsection}{\normalfont\large\bfseries}{\thesubsection}{1em}{}
\titleformat{\subsubsection}{\normalfont\normalsize\bfseries}{\thesubsubsection}{1em}{}
\titleformat{\paragraph}{\normalfont\normalsize\itshape\bfseries}{{\normalfont\normalsize\bfseries\theparagraph}}{1em}{}

% Add new sublevel
\newcommand{\subsubsubsection}[1]{\paragraph{#1}\mbox{}\\\vspace{-5em}}
% A note - this command only works well if there is an empty line after it


\newcommand{\references}{
    \fancyhead[L]{\headercase{References}}
    \addcontentsline{toc}{section}{References}
    \setlength\bibitemsep{1.5\itemsep}
    \defbibheading{SpecialBibHeadingStyle}{\section*{References}}
    \printbibliography[heading=SpecialBibHeadingStyle]
}



% Use at start of appendix section
\newcommand{\appendices}{
    \appendix
    \titleformat{\section}{\normalfont\Large\bfseries}{\thesection}{1em}{}
    \fancyhead[L]{\headercase{Appendices}}

    \section*{Appendices}
    \addtocounter{section}{1}
    \addcontentsline{toc}{section}{Appendices}

    \titleformat{\subsection}{\normalfont\Large\bfseries}{\thesubsection}{1em}{}
    \titleformat{\subsubsection}{\normalfont\large\bfseries}{\thesubsubsection}{1em}{}
    \titleformat{\paragraph}{\normalfont\normalsize\bfseries}{\paragraph}{1em}{}
}







% ---------------------------------------------------------------------------------------------------
% || ----------------------------------- TITLEPAGE COMMANDS -------------------------------------- ||
% ---------------------------------------------------------------------------------------------------

\newcommand{\topTitle}{
    {
        \centering

        {\LARGE \@title}

        \if \@author \empty
        \else
            {\large \@author}\\
        \fi
    }
    \vspace{3em}
}




\newcommand{\fullPageTitle}{
    \begin{titlepage}

        \vspace{\stretch{1}}

        \newcommand{\HRule}{\rule{\linewidth}{0.5mm}} % Defines a new command for the horizontal lines, change thickness here
        
        \center % Center everything on the page
         

        % HEADINGS -----------------------------------------------------------------------------------
        \vspace*{\fill}
        \if \Uni \empty
        \else
            {\scshape\huge ~\Uni}\\[1em] % Name of your university/college
        \fi
        
        \if \School \empty
        \else
            {\scshape\Large ~\School}\\[4em] % Major heading such as course name
        \fi
        
        \if \Unit \empty
        \else
            {\scshape\Large ~\Unit}\\[1em]
        \fi

        \if \Class \empty
        \else
            {\scshape\large ~\Class}\\[3em]% Minor heading such as course title
        \fi


        % TITLE -----------------------------------------------------------------------------------
        \HRule \\[0.4cm]
        { \Huge \bfseries ~\@title}\\[0.1cm] % Title of your document
        \HRule \\[4em]
         

        % AUTHOR ----------------------------------------------------------------------------------
        \if \@author \empty
        \else
            {\large \@author}
        \fi
        

        \vspace{\stretch{4}} % Fill the rest of the page with whitespace

        % DATE ----------------------------------------------------------------------------------------
        \if \@date \empty
        \else
            {\large ~\@date}\\ % Date, change the \today to a set date if you want to be precise
            \vspace{4em}
        \fi
        
        
        
        
        
        \end{titlepage}
}




% ---------------------------------------------------------------------------------------------------
% || --------------------------------- MISC. DOCUMENT FORMATTING --------------------------------- ||
% ---------------------------------------------------------------------------------------------------

\let\cleardoublepage=\clearpage % stops random pages being printed after the ToC and titlepage

% Sets the maximum number of levels in the Table of Contents to x-1
\setcounter{secnumdepth}{4}


\newcommand{\super}{\textsuperscript}
\newcommand{\sub}{\textsubscript}


% Creates new column type for tables
\newcolumntype{C}[1]{>{\centering\arraybackslash}m{#1}}

% Sets the folder path for images to be a subfolder of where the .tex file is
\graphicspath{{./images/}}

% Headers and footers
\pagestyle{fancy}
\fancyhf{}
\setlength{\headheight}{15.2pt}

% overridden by customisation below
\fancyhead[L]{\headercase{\leftmark}} % Left header - last section on page
\fancyhead[R]{\SID} % Right Header
\fancyfoot[C]{\small\thepage} % Puts page number bottom centre


\fancypagestyle{noheader}{
  \fancyhf{}% Clear header/footer
  \renewcommand{\headrulewidth}{0pt}% No header rule
  \fancyfoot[RE, RO]{\thepage}
}



% Equation Explanation
\newcommand{\eqexp}[1]{
    \begin{adjustwidth}{0.5cm}{1cm}
        \textit{#1}
    \end{adjustwidth}
    \vspace{0.5em}
}









% -------------------------------------------------------------
%   --- New LaTeX users feel free to edit what is below ---   |
% -------------------------------------------------------------

% Making any of these fields blank will remove them from the titlepage
\newcommand{\Uni}{The University of Sydney}
\newcommand{\School}{School of Aerospace Mechanical and Mechatronics Engineering}
\newcommand{\Unit}{Unit Code}
\newcommand{\Class}{Class Time}
\newcommand{\Assignment}{Assignment Name}
\newcommand{\SID}{123456789}

\author{Author 1 \\[0.5em] Author 2 \\[0.5em] Author 3 \\}

\date{} % makes it undated, remove this line if you want a date to appear
%\date{\today} % make the date the current date





% ----------------------------------------------------------------------
%   --- Semi-Advanced LaTeX users feel free to edit what is below ---  |
% ----------------------------------------------------------------------

\title{\Assignment}

% Formatting
\numberwithin{equation}{section} % numbers equation with secNum.EqNum, remove to number globally

\setlength{\parindent}{0em} % indent when there is a double newline within the tex code
\setlength{\parskip}{1em} % vertical space when there is a double newline within the text code


\usepackage{hyperref,xcolor}
\hypersetup{
    colorlinks=true,
    linkcolor=darkblue,
    filecolor=magenta,      
    urlcolor=blue,
    citecolor=black
}


\newcommand{\headercase}[1]{\scshape\nouppercase{#1}}
\fancyhead[L]{\headercase{\leftmark}} % Left header - last section on page
\fancyhead[R]{\SID} % Right Header
\fancyfoot[C]{\small\thepage} % Puts page number bottom centre




% -----------------------------------------------------------------
%   --- Advanced LaTeX users... do what you want ¯\_(ツ)_/¯ ---   |
% -----------------------------------------------------------------



\begin{document}
    \pagenumbering{gobble} % turns off page numbering

    \fullPageTitle
    %\topTitle

    \thispagestyle{noheader}
    \tableofcontents

    % \vspace{2em}
    % \listoffigures

    % \vspace{2em}
    % \listoftables

    \newpage
    \pagenumbering{arabic} % restarts page numbering in arabic numerals

    \section{What This Is}
        \label{sec: what this is}
        
        Here I've used the \verb+Report Template.tex+ document to create a guide to said document. Below is a guide to how to edit the \LaTeX~if you are new, as well as the different commands that have been added on top of the default \LaTeX~commands.
    

    \section{Editing the Preamble and Titlepage}
        \label{sec: editing the preamble}

        To edit the preamble (provided you're somewhat new to \LaTeX) scroll to the bottom of \\\verb+Report Preamble.tex+ until you see the comment:

        \begin{lstlisting}
% -------------------------------------------------------------
%   --- New LaTeX users feel free to edit what is below ---   |
% -------------------------------------------------------------
        \end{lstlisting}



        \subsection{The Parameters and What They Do}
            \label{subsec: the parameters and what they do}

            \verb+\newcommand{\Uni}{The University of Sydney}+\\
            This command edits part of the header of the titlepage. Edit the second field to change the header, making it empty removes that part of the header.

            \verb+\newcommand{\School}{School of Aerospace Mechanical and Mechatronics Engineering}+\\
            This command edits part of the header of the titlepage. Edit the second field to change the header, making it empty removes that part of the header.

            \verb+\newcommand{\Unit}{Unit Code}+\\
            This command edits part of the body of the titlepage. Edit the second field to change the text, making it empty removes that part of the titlepage.

            \verb+\newcommand{\Class}{Class Time}+\\
            This command edits part of the body of the titlepage. Edit the second field to change the text, making it empty removes that part of the titlepage.

            \verb+\newcommand{\Assignment}{Assignment Name}+\\
            This changes the title of the titlepage. Edit the second field to change the text.

            \verb+\newcommand{\SID}{123456789}+\\
            This edits your student ID that appears in the headers.

            \verb+\author{Author 1 \\[0.5em] Author 2 \\[0.5em] Author 3 \\}+\\
            This edits the author field on the titlepage. To add a new author you can add a \verb+\\+ between names (or just a comma). To increase the spacing change it to a \verb+\\[<extra spacing>]+ e.g. \verb+\\[2mm]+.

            \verb+\date{}+\\
            This adds the date to the titlepage. If the field is empty it doesn't add it to the page. Making the field \verb+\today+ automatically changes the field to the date of compilation.

            \newpage

            %More complex commands
            \verb+\title{\Assignment}+\\
            Changes the title field. By default uses the \verb+\Assignment+ macro but can be edited to add extra text. (\LaTeX~tries to outsmart you if you add text after a command by removing the space between the command text and the new text, use a ``\verb+~+'' e.g. \verb+\Assignment~extra text+ to force it to add a space.)

            \verb+\numberwithin{equation}{section}+\\
            Changes the numbering of equations. As above it numbers them as (\verb+<sec num>.<eq num>+) where the equation number counter gets reset at the start of each section. To just number them by equation (ignoring sections) make the second field empty.

            \verb+\setlength{\parindent}{0em}+\\
            This sets the indent at the start of a paragraph. \LaTeX~treats a new paragraph as starting whenever within your \verb+.tex+ file there is an empty line followed by new text. By default \LaTeX~indents the first line of this new text, changing the second field changes this indent.

            \verb+\setlength{\parskip}{1em}+\\
            This sets the length of a space between paragraphs. LaTeX treats a new paragraph as starting whenever within your \verb+.tex+ file there is an empty line followed by new text. By default \LaTeX~doesn't add any space between paragraphs, changing the second field changes the spacing between paragraphs.

            \verb+\newcommand{\headercase}[1]{\scshape\nouppercase{#1}}+\\
            Macro for controlling the header style, takes the header text as its argument. \verb+\scshape+ sets it to small caps, you can change this or add multiple styles. Other options include \verb+\itshape+ ({\itshape italic}), \verb+\bfseries+ ({\bfseries bold}) and \verb+\slshape+ ({\slshape slanted}).\\
            Remove the \verb+\nouppercase+ command to force it to all uppercase.

            \verb+\fancyhead[L]{\headercase{\leftmark}}+\\
            This changes the left side of the header. \verb+\leftmark+ contains the last section that is started on a given page, \verb+\rightmark+ contains the last heading (i.e. including subsections etc.) that is started on a given page.

            \verb+\fancyhead[R]{\SID}+\\
            Changes the right side of the header. By default uses the custom macro \verb+\SID+ but can be changed.

            \verb+\fancyfoot[C]{\small\thepage}+\\
            Changes the centre of the footer, by default adds the page number with \verb+\thepage+.

    \newpage

    \section{The Custom Commands / Macros}
        \label{sec: custom commands}

        \subsection{Cross Referencing Other Sections}
            \label{subsec: cross referencing}

            Internal cross referencing allows you to refer the reader to another part of the document using the corresponding \verb+\label+. The default \LaTeX~commands are not terrible but by default it only generates a hyperlinked form of the number of whatever you're referring to. Instead we can use the commands below.

            \verb+\secref{<label>}+\\
            This references a section by number and name e.g. \secref{sec: editing the preamble}.\\
            This can also be changed to just reference by just number or name by changing the line in the preamble from \verb+\newcommand{\secref}[1]{\secrefNumName{#1}}+ to:\\
            \verb+\newcommand{\secref}[1]{\secrefNum{#1}}+ $\rightarrow$ \secrefNum{sec: editing the preamble}\\
            \verb+\newcommand{\secref}[1]{\secrefName{#1}}+ $\rightarrow$ \secrefName{sec: editing the preamble}
            
            \verb+\figref{<label>}+ or \verb+\Figref{<label>}+\\
            This generates a cross reference to a figure, \verb+\Figref+ capitalises the word ``Figure''. E.g. there is \figref{fig: latex logo} or \Figref{fig: latex logo}.

            \verb+\eqref{<label>}+ or \verb+\Eqref{<label>}+\\
            This generates a cross reference to an equation, \verb+\Eqref+ capitalises the word ``Equation''. E.g. there is \eqref{eq: Gauss Law} or \Eqref{eq: Gauss Law}.

            \verb+\tableref{<label>}+ or \verb+\Tableref{<label>}+\\
            This generates a cross reference to a table, \verb+\Table+ capitalises the word ``Table''. E.g. there is \tableref{table: example} or \Tableref{table: example}.

            \verb+\appref{<label>}+ or \verb+\Appref{<label>}+\\
            This generates a cross reference to an appendix, \verb+\Appref+ capitalises the word ``Appendix''. E.g. there is \appref{app: example figure} or \Appref{app: example figure}.

            \verb+\eqnum{<label>}+\\
            Just grabs the number of an equation. If you want to repeat an equation but keep the original number you can use \verb+\tag{\eqnum{<label>}}+ where \verb+<label>+ is the label of the original equation.

        \newpage

        \subsection{Creating References and Appendices}
            \label{subsec: creating references and appendices}

            To streamline creating the references list and appendices there are two commands:

            \verb+\references+\\
            This prints the list of (cited\footnote{\LaTeX~uses the \BibTeX~engine to generate references. The advantage is you can have a big file of references and it does all the styling for you, the downside is that it only adds to the referencing list the ones that you in-text cite.}) references and adds a line to the table of contents for the references. To change the referencing style change the line \\\verb+\usepackage[backend=biber, style=ieee]{biblatex}+. See \textcite{referencing-styles} for other styles.

            \verb+\appendices+\\
            This starts the appendices section, changes the header and reformats the titles so that subsections look like sections etc. This is because to get the numbering correct each appendix has to be a subsection.


        \subsection{Title Page Commands}
            \label{subsec: title page commands}

            There are two titlepage commands that can be used:

            \verb+\fullPageTitle+\\
            Creates full title page as is in this document. To customise the titlepage edit the parameters in the preamble as per \secref{subsec: the parameters and what they do}.

            \verb+\topTitle+\\
            Creates a title at the top of the page. This is intended for scientific reports which are typically more understated with an abstract below the title.


        \subsection{Extra Section Command}
            \label{subsec: extra section command}

            Typically \LaTeX~only allows 3 levels of sections: \verb+\section+, \verb+\subsection+ and \verb+\subsubsection+. I have added a 4\super{th} command \verb+\subsubsubsection+.


        \subsection{Superscript and Subscript}
            \label{subsec: superscript and subscript}

            Shortcuts for superscript and subscript in text \verb+\super{}+ and \verb+\sub{}+.\\
            E.g. \verb+1\super{st}+ $\to$ 1\super{st} and \verb+12\sub{dec}+ $\to$ 12\sub{dec}.


        \subsection{Maths Commands}
            \label{subsec: maths commands}

            \verb+\Reals+ $\Reals$
            
            \verb+\Complexs+ $\Complexs$
            
            \verb+\Integers+ $\Integers$
            
            \verb+\Naturals+ $\Naturals$
            
            \verb+\Rationals+ $\Rationals$ 

            \verb+\Primes+ $\Primes$

            \verb+\emf+ $\emf$

            \verb+\deq+ $\deq$

            \verb+\d+ \footnote{overrides underdot accent.}\\
            A different font `d' for integrals.
            \begin{equation*}
                x = \int v_x \d x
            \end{equation*}

            \verb+\ud+\\
            A different font `d' for integrals with a space.
            \begin{equation*}
                x = \int v_x \ud x
            \end{equation*}

            \verb+\bfrac{}+ and \verb+\bint{}+\\
            Commands that force sizing of fractions and integrals to be big.
            
            \verb+\vect{}+ and \verb+\vhat{}+\\
            Custom vector formatting e.g. $\vect{v}$ and $\vhat{v}$.

            \verb+\hvec{}+\\
            Vector with harpoon accent e.g. $\hvec{v}$

            \verb+\svec{}+\\
            Vector with squiggle accent e.g. $\svec{v}$

            \verb+\bvec{}+\\
            Vector with bold text e.g. $\bvec{v}$.

            \verb+\lhvec{}+ and \verb+\lvect{}+\\
            A long vector with harpoon accent e.g. $\lhvec{AB}$ and $\lvect{AB}$.

            \verb+\vmod{}+\\
            Vector modulus notation e.g. \verb+\vmod{x}+ = $\vmod{x}$.

            \verb+\xoverline{<x>}+ or \verb+\xoverline[<width fraction>]{<x>}+\\
            Places a line over \verb+<x>+ with optional argument \verb+<width fraction>+ e.g. \verb+\xoverline{x}+ = $\xoverline{x}$ or \verb+\xoverline[0.5]{average}+ = $\xoverline[0.5]{average}$.

            \verb+\Matrix{}+\\
            Shortcut for matrix environment surrounded by square brackets e.g. \verb+\Matrix{1 & 2 \\ 3 & 4}+
            \begin{equation*}
                \Matrix{1 & 2 \\ 3 & 4}
            \end{equation*}

            \verb+\proj{a}{b}+\\
            Projection of $a$ onto $b$ e.g. $\proj{a}{b}$


        

    \newpage

    \pagenumbering{roman}


    \references

    \newpage


    \appendices

        \subsection{Example Figure}
            \label{app: example figure}

            \begin{figure}[!h]
                \centering
                \includegraphics[width=0.5\textwidth]{LaTeX Logo.png}    

                \caption{The \LaTeX~logo. Figure captions typically go below a figure.}
                \label{fig: latex logo}
            \end{figure}

            \FloatBarrier

        \subsection{Example Table}
            \label{app: example table}

            \begin{table}[!h]
                \centering
                \caption{An example table. Table captions typically go above a table.}
                \label{table: example}
                \begin{tabular}{ |C{0.2\textwidth}|m{0.2\textwidth}| }
                    \hline
                    Test & text\\
                    \hline
                    \hline
                    Test & text\\
                    \hline
                    Test & text\\
                    \hline
                \end{tabular}
            \end{table}

        \subsection{Example Equation}
            \label{app: example equation}

            \begin{equation}
                \oiint_S \vect{E} \cdot \d\vect{A} = \frac{Q_{\text{enc}}}{\varepsilon_0}
                \label{eq: Gauss Law}
            \end{equation}

            
            
\end{document}